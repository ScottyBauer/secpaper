\documentclass{sig-alternate-05-2015}


\begin{document}

% Copyright
\setcopyright{acmcopyright}
%\setcopyright{acmlicensed}
%\setcopyright{rightsretained}
%\setcopyright{usgov}
%\setcopyright{usgovmixed}
%\setcopyright{cagov}
%\setcopyright{cagovmixed}


%
% --- Author Metadata here ---
%\CopyrightYear{2007} % Allows default copyright year (20XX) to be over-ridden - IF NEED BE.
%\crdata{0-12345-67-8/90/01}  % Allows default copyright data (0-89791-88-6/97/05) to be over-ridden - IF NEED BE.
% --- End of Author Metadata ---

\title{Mitigating Sigreturn Oriented Programming Exploits with Dessert}
\subtitle{[Extended Abstract]
\titlenote{A full version of this paper is available as
\textit{Author's Guide to Preparing ACM SIG Proceedings Using
\LaTeX$2_\epsilon$\ and BibTeX} at
\texttt{www.acm.org/eaddress.htm}}}
%
% You need the command \numberofauthors to handle the 'placement
% and alignment' of the authors beneath the title.
%
% For aesthetic reasons, we recommend 'three authors at a time'
% i.e. three 'name/affiliation blocks' be placed beneath the title.
%
% NOTE: You are NOT restricted in how many 'rows' of
% "name/affiliations" may appear. We just ask that you restrict
% the number of 'columns' to three.
%
% Because of the available 'opening page real-estate'
% we ask you to refrain from putting more than six authors
% (two rows with three columns) beneath the article title.
% More than six makes the first-page appear very cluttered indeed.
%
% Use the \alignauthor commands to handle the names
% and affiliations for an 'aesthetic maximum' of six authors.
% Add names, affiliations, addresses for
% the seventh etc. author(s) as the argument for the
% \additionalauthors command.
% These 'additional authors' will be output/set for you
% without further effort on your part as the last section in
% the body of your article BEFORE References or any Appendices.

\numberofauthors{2} %  in this sample file, there are a *total*
% of EIGHT authors. SIX appear on the 'first-page' (for formatting
% reasons) and the remaining two appear in the \additionalauthors section.
%
\author{
% You can go ahead and credit any number of authors here,
% e.g. one 'row of three' or two rows (consisting of one row of three
% and a second row of one, two or three).
%
% The command \alignauthor (no curly braces needed) should
% precede each author name, affiliation/snail-mail address and
% e-mail address. Additionally, tag each line of
% affiliation/address with \affaddr, and tag the
% e-mail address with \email.
%
% 1st. author
\alignauthor
Scott Bauer
       \email{sbauer@eng.utah.edu}
% 2nd. author
\alignauthor
Abhiram Balasubramanian
       \email{abhiram@cs.utah.edu}
% Just remember to make sure that the TOTAL number of authors
% is the number that will appear on the first page PLUS the
% number that will appear in the \additionalauthors section.

\maketitle
\begin{abstract}
As exploitation mitigations advance attackers look for new methods to control the flow of a program after exploiting a vulnerability. Recent exploit mitigations such as StackCookies [CITE ME], Data Execution Prevention (W^X) [CITE ME] and Address Space Layout Ranomization (ASLR) [CITE ME], have forced exploit authors to use techniques that jump around the text segment of an ELF in memory to achieve code execution. The newest form of exploitation, Sigreturn Oriented Programming (SROP), uses signal frames, and the kernel to achieve code execution by tricking the kernel to change the instruction pointer to some arbitrary location in the code.\\
Signals are one of the commonly used kernel to user/user to user asynchronous notification mechanisms in UNIX. By registering a signal handler, a process can handle these asynchronous events outside its control flow. As signal frames are stored on the process stack, attackers can create exploits by setting up dummy signal frames on the stack and initiate returns from signals that the kernel never delivered. This method bypasses all current exploit mitigation techniques and  appears more convenient as kernel never keeps track of the signals it delivered.\\
We develop a mechanism on top of the 4.1 Linux kernel to systematically stop SROP exploitaion.

\end{abstract}


%
% The code below should be generated by the tool at
% http://dl.acm.org/ccs.cfm
% Please copy and paste the code instead of the example below. 
%
\begin{CCSXML}
\end{CCSXML}

\ccsdesc[500]{Computer systems organization~Embedded systems}
\ccsdesc[300]{Computer systems organization~Redundancy}
\ccsdesc{Computer systems organization~Robotics}
\ccsdesc[100]{Networks~Network reliability}


%
% End generated code
%

%
%  Use this command to print the description
%
\printccsdesc

% We no longer use \terms command
%\terms{Theory}

\keywords{ACM proceedings; \LaTeX; text tagging}

\section{Introduction}
Over the past 25 years Operating System designers and compiler designers have been facing off against exploit developers. The OS and Compiler designers  come up with a novel technique to suppress some form of exploitation method, for example Stack cookies, ASLR, DEP, etc. The exploit developers however continuously find methods around these exploit mitigations and the cat-and-mouse game continues.  A paper titled “Framing Signals -- A return to portable Shellcode”,  presents a new novel exploitation technique based off the idea of Return Oriented Programming (ROP). In a ROP an attacker locates gadgets in the assembly and continuously returns to those locations to build working shellcode. In this new technique attackers leverage the way Unix-like operating systems deliver signals to applications to create portable and reliable shellcode. This technique works by crafting a signal frame on the stack and tricking the kernel into thinking it is a legitimate signal frame. This causes the kernel to tell the process to start executing where the attacker wants leading to arbitrary code execution.\\
\indent
To mitigate this new style of exploitation we design, implement and use a SROP mitigation technique based off of the idea of cookies originally developed to stop stack based buffer overflows. We test our implementation against two different SROP exploits, both of which are stopped by the new mitigation technique.



\section{The {\secit Body} of The Paper}
\subsection{Type Changes and {\subsecit Special} Characters}


\subsection{Figures}
Like tables, figures cannot be split across pages; the
best placement for them
is typically the top or the bottom of the page nearest
their initial cite.  To ensure this proper ``floating'' placement
of figures, use the environment
\textbf{figure} to enclose the figure and its caption.

This sample document contains examples of \textbf{.eps} files to be
displayable with \LaTeX.  If you work with pdf\LaTeX, use files in the
\textbf{.pdf} format.  Note that most modern \TeX\ system will convert
\textbf{.eps} to \textbf{.pdf} for you on the fly.  More details on
each of these is found in the \textit{Author's Guide}.

\begin{figure}
\centering
\includegraphics{fly}
\caption{A sample black and white graphic.}
\end{figure}

\begin{figure}
\centering
\includegraphics[height=1in, width=1in]{fly}
\caption{A sample black and white graphic
that has been resized with the \texttt{includegraphics} command.}
\end{figure}


As was the case with tables, you may want a figure
that spans two columns.  To do this, and still to
ensure proper ``floating'' placement of tables, use the environment
\textbf{figure*} to enclose the figure and its caption.
and don't forget to end the environment with
{figure*}, not {figure}!

\begin{figure*}
\centering
\includegraphics{flies}
\caption{A sample black and white graphic
that needs to span two columns of text.}
\end{figure*}


\begin{figure}
\centering
\includegraphics[height=1in, width=1in]{rosette}
\caption{A sample black and white graphic that has
been resized with the \texttt{includegraphics} command.}
\vskip -6pt
\end{figure}

\subsection{Theorem-like Constructs}
Other common constructs that may occur in your article are
the forms for logical constructs like theorems, axioms,
corollaries and proofs.  There are
two forms, one produced by the
command \texttt{{\char'134}newtheorem} and the
other by the command \texttt{{\char'134}newdef}; perhaps
the clearest and easiest way to distinguish them is
to compare the two in the output of this sample document:

This uses the \textbf{theorem} environment, created by
the\linebreak\texttt{{\char'134}newtheorem} command:
\newtheorem{theorem}{Theorem}
\begin{theorem}
Let $f$ be continuous on $[a,b]$.  If $G$ is
an antiderivative for $f$ on $[a,b]$, then
\begin{displaymath}\int^b_af(t)dt = G(b) - G(a).\end{displaymath}
\end{theorem}

The other uses the \textbf{definition} environment, created
by the \texttt{{\char'134}newdef} command:
\newdef{definition}{Definition}
\begin{definition}
If $z$ is irrational, then by $e^z$ we mean the
unique number which has
logarithm $z$: \begin{displaymath}{\log e^z = z}\end{displaymath}
\end{definition}

Two lists of constructs that use one of these
forms is given in the
\textit{Author's  Guidelines}.
 
There is one other similar construct environment, which is
already set up
for you; i.e. you must \textit{not} use
a \texttt{{\char'134}newdef} command to
create it: the \textbf{proof} environment.  Here
is a example of its use:
\begin{proof}
Suppose on the contrary there exists a real number $L$ such that
\begin{displaymath}
\lim_{x\rightarrow\infty} \frac{f(x)}{g(x)} = L.
\end{displaymath}
Then
\begin{displaymath}
l=\lim_{x\rightarrow c} f(x)
= \lim_{x\rightarrow c}
\left[ g{x} \cdot \frac{f(x)}{g(x)} \right ]
= \lim_{x\rightarrow c} g(x) \cdot \lim_{x\rightarrow c}
\frac{f(x)}{g(x)} = 0\cdot L = 0,
\end{displaymath}
which contradicts our assumption that $l\neq 0$.
\end{proof}

Complete rules about using these environments and using the
two different creation commands are in the
\textit{Author's Guide}; please consult it for more
detailed instructions.  If you need to use another construct,
not listed therein, which you want to have the same
formatting as the Theorem
or the Definition\cite{salas:calculus} shown above,
use the \texttt{{\char'134}newtheorem} or the
\texttt{{\char'134}newdef} command,
respectively, to create it.

\subsection*{A {\secit Caveat} for the \TeX\ Expert}
Because you have just been given permission to
use the \texttt{{\char'134}newdef} command to create a
new form, you might think you can
use \TeX's \texttt{{\char'134}def} to create a
new command: \textit{Please refrain from doing this!}
Remember that your \LaTeX\ source code is primarily intended
to create camera-ready copy, but may be converted
to other forms -- e.g. HTML. If you inadvertently omit
some or all of the \texttt{{\char'134}def}s recompilation will
be, to say the least, problematic.

\section{Conclusions}
This paragraph will end the body of this sample document.
Remember that you might still have Acknowledgments or
Appendices; brief samples of these
follow.  There is still the Bibliography to deal with; and
we will make a disclaimer about that here: with the exception
of the reference to the \LaTeX\ book, the citations in
this paper are to articles which have nothing to
do with the present subject and are used as
examples only.
%\end{document}  % This is where a 'short' article might terminate

%ACKNOWLEDGMENTS are optional
\section{Acknowledgments}
This section is optional; it is a location for you
to acknowledge grants, funding, editing assistance and
what have you.  In the present case, for example, the
authors would like to thank Gerald Murray of ACM for
his help in codifying this \textit{Author's Guide}
and the \textbf{.cls} and \textbf{.tex} files that it describes.

%
% The following two commands are all you need in the
% initial runs of your .tex file to
% produce the bibliography for the citations in your paper.
\bibliographystyle{abbrv}
\bibliography{sigproc}  % sigproc.bib is the name of the Bibliography in this case
% You must have a proper ".bib" file
%  and remember to run:
% latex bibtex latex latex
% to resolve all references
%
% ACM needs 'a single self-contained file'!
%
%APPENDICES are optional
%\balancecolumns
\appendix
%Appendix A
\section{Headings in Appendices}
The rules about hierarchical headings discussed above for
the body of the article are different in the appendices.
In the \textbf{appendix} environment, the command
\textbf{section} is used to
indicate the start of each Appendix, with alphabetic order
designation (i.e. the first is A, the second B, etc.) and
a title (if you include one).  So, if you need
hierarchical structure
\textit{within} an Appendix, start with \textbf{subsection} as the
highest level. Here is an outline of the body of this
document in Appendix-appropriate form:
\subsection{Introduction}
\subsection{The Body of the Paper}
\subsubsection{Type Changes and  Special Characters}
\subsubsection{Math Equations}
\paragraph{Inline (In-text) Equations}
\paragraph{Display Equations}
\subsubsection{Citations}
\subsubsection{Tables}
\subsubsection{Figures}
\subsubsection{Theorem-like Constructs}
\subsubsection*{A Caveat for the \TeX\ Expert}
\subsection{Conclusions}
\subsection{Acknowledgments}
\subsection{Additional Authors}
This section is inserted by \LaTeX; you do not insert it.
You just add the names and information in the
\texttt{{\char'134}additionalauthors} command at the start
of the document.
\subsection{References}
Generated by bibtex from your ~.bib file.  Run latex,
then bibtex, then latex twice (to resolve references)
to create the ~.bbl file.  Insert that ~.bbl file into
the .tex source file and comment out
the command \texttt{{\char'134}thebibliography}.
% This next section command marks the start of
% Appendix B, and does not continue the present hierarchy
\section{More Help for the Hardy}
The sig-alternate.cls file itself is chock-full of succinct
and helpful comments.  If you consider yourself a moderately
experienced to expert user of \LaTeX, you may find reading
it useful but please remember not to change it.
%\balancecolumns % GM June 2007
% That's all folks!
\end{document}
